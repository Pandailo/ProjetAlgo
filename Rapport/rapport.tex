\documentclass[a4paper]{report}


\usepackage{geometry}         % marges
\usepackage[francais]{babel}  % langue
\usepackage[utf8]{inputenc}
\usepackage[T1]{fontenc}
\usepackage{graphicx}         % images
\usepackage{verbatim}         % texte préformaté
\usepackage{url}			  % package url (voir bibli)
\usepackage{lmodern}
\usepackage{amsmath}
\usepackage{amssymb}
\usepackage{mathrsfs}
\title{Rapport de projet UE Algorithmique et complexité}
\author{Yann VERON}
\author{Marvin NURIT}
\date{12 Décemnbre 2016}

\pagestyle{headings}

\begin{document}

%\setcounter{page}{0} % enlever numéro de page
\thispagestyle{empty}
 
\begin{minipage}{0.5\linewidth} % permet de faire les deux colonnes pour mettre l'image à droite du texte
\textbf{Université de Bourgogne}\\ 
U.F.R. Sciences et Technologies\\
Département Informatique\\
Année 2016-2017
\end{minipage}
\begin{minipage}{0.5\linewidth}
\begin{flushright}
\includegraphics{ub.jpg}
\end{flushright}

\end{minipage}
 
 
\vspace{5cm}
 
 
\begin{center}
\LARGE Master \\
\LARGE M1 Informatique\\
\huge \textbf{RAPPORT DE PROJET}\\
\huge \textbf{UE\\ Algorithme et Complexité}\\
\end{center}
 
 
\vspace{7cm}
 
 
\noindent % permet d'enlever l'indentation du paragraphe
\textbf{Etudiants  :} Yann Véron et Marvin Nurit \\
\textbf{Sujet :} Multiplication rapide de grands entiers \\

\tableofcontents
% --- inserer votre code LaTeX ici ---
\chapter{Problématique}

  \section{Sujet}

    \subsection{Description générale} %Grande lignes du sujet
      \paragraph{}
      Dans ce sujet, il nous est demandé de programmer en langage OCaml la multiplication de grands entier de deux façons différentes, l'algorithme de Karatsuba \cite{KaratsuINRIA} \cite{KaratsuUnivLyon} \cite{KaratsuWiki}  et la Transformée de Fourier Rapide.\\
      La multiplication de deux entiers de façon "naïve" de taille n se fait avec $n^2$ multiplication, ce qui correspond à une complexité de l'ordre de $O(n)$.
      \\Cette complexité, et donc le temps d'exécution de cette multiplication peuvent poser problème sur de grands entiers et polynômes.
      \\C'est pourquoi les deux algorithmes cités précédemment, Fourier et Karatsuba, ont été mis au point.
	%Description précise de chaque fonction
      \paragraph{}
      	\subsection{Multiplication par l'algorithme de Karatsuba} 
      	Anatoli Alekseïevitch Karatsuba est un mathématicien Russe ayant obtenus plusieurs récompenses pour ses travaux sur l'algorithmique Multiprécision, plus particulièrement la multiplication.
      	\\ Il donne ainsi son nom au premier algorithme de multiplication rapide, l'algorithme de Karatsuba, à démontré un théorème d'approximation des séries de Fourier, et amélioré le théorème de Moore sur la machine de Moore.
      	\\ Dans notre cas, c'est son algorithme de multiplication rapide qui nous intéresse.
      	\\ Comme dit précédemment, la multiplication "naïve" à une complexité d'ordre $O(n)$, tandis que l'algorithme de Karatsuba à une complexité d'ordre au plus $n^{log_2 3}$ $\approx$ $n^{1.585}$, comme nous le verrons prochainement.
      	\\ L'algorithme de Karatsuba est de type "Divide and Conquer", ou "Diviser pour régner".
      	\\ L'objectif est de décomposer les deux entiers en deux plus petit (dont la taille est approximativement égale à la moitié de la taille des deux grands entiers), puis de faire la multiplication des ces deux plus petits entiers.
      	\\ La méthode de Karatsuba permet de faire cette dernière multiplication en 3 produits au lieu de 4 de façon naïve. Pour de plus grands entiers il suffira d'appliquer récursivement cette méthode.
      \paragraph{}  
    	\subsection{Multiplication par la transformée de Fourier Rapide} 

      
\chapter{Choix d'implémentation}

  \section{Algorithmes}   
 	\paragraph{}
      	\subsection{Implémentation des polynômes}
      	Nous avons choisi de représenter nos grands entiers en base $B$ en polynômes.
      	\\ Ces polynômes seront représentés en OCaml sous la forme de listes, grâce au module List implémenté dans ce langage.
      	\\ Les entiers seront représentés bit de poids faible en premier pour faciliter la manipulation.
      	\\ Ainsi un entier 1234 en base 10, dont le polynômes vaut $1\times{10^3}+2\times{10^2}+3\times{10^1}+4\times{10^0}$ sera représenté par la liste $L=[4;3;2;1]$.
      	\\ De même pour une base 16, l'entier 26, codé de manière hexadécimale 1A, et dont le polynôme vaut $1\times16^1+11\times16^0$ sera représenté par la liste $L=[11,1]$.
      	\\ Pour cette conversion, et l'utilisation des listes dans les fonctions suivantes il faudra de ce fait passer en paramètre la base sur laquelle le polynôme à été créé, la liste pouvant représenter différents polynômes différents selon la base.
      	\\ La reconstruction de l'entier depuis la liste fonctionne de la même manière :
      	\\ En fonction de la base passée en paramètre on additionne les membres du polynôme que l'on multiplie par la base à la puissance adéquate.
      	\\Pour plus de facilité d'utilisation, il aurait sans doute été possible de mettre en premier la base dans laquelle la liste à été créé.
      \paragraph{}
      	\subsection{Algorithme de Karatsuba} 
      	
      \paragraph{}  
    	\subsection{Transformée de Fourier Rapide} 
    	
      \paragraph{}
       
\bibliographystyle{plain} % Le style est mis entre accolades.
\bibliography{bibli} % mon fichier de base de données s'appelle bibli.bib
\end{document}